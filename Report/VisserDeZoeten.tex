
\documentclass[12pt]{amsart}
\usepackage{geometry} % see geometry.pdf on how to lay out the page. There's lots.
\geometry{a4paper} % or letter or a5paper or ... etc
% \geometry{landscape} % rotated page geometry

% See the ``Article customise'' template for come common customisations

\title{Computer Vision}
\author{Anouk Visser and R\'emi de Zoeten}
\date{} % delete this line to display the current date

%%% BEGIN DOCUMENT
\begin{document}

\maketitle
%\tableofcontents

\section{Assignment 1}
\subsection{ICP}
There are different ways to improve the efficiency and the accuracy of ICP. Some of these techniques are discussed in ``Efficient variants of the ICP algorithm.''. In this assignment, students should also analyze various aspects such as accuracy, speed, stability and tolerance to noise by changing the point selection technique. Using all the points, uniform sub-sampling, random sub-sampling in each iteration and sub-sampling more from informative regions can be used as the point selection technique. Students are expected to implement these variants and report their findings in the final report.
\subsection{Merging scenes}
Estimate the camera poses using each two consecutive frames of given data. Using the estimated camera poses, merge the point-clouds of all the scenes into one point-cloud and visualize the result. 
\\\\
\textbf{Does the merging produce sufficient result? Discuss why.}
\\\\
\textbf{Now, estimate the camera pose and merge the results using every 2nd, 4th, and 10th frames. Does the camera pose estimation change?}
\\\\
Iteratively merge and estimate the camera poses for the consecutive frames (point-clouds). Supposed there are N frames (frame1, frame2 ,. . ., frameN). First, estimate the camera pose of frame2 (target) from frame1(base). Next, merge frame1 and frame2 into frame1,2. Then, estimate the camera pose of frame3(target) from frame1,2(base) and use it to merge frame1,2 and frame3 into frame1,2,3 and so on until reaching the final frame (frameN ).\\\\ \textbf{Do the estimated camera poses change in comparison with the previous estimates (Section 2.1)? Does this estimation produce better results?}
\subsection{Questions}
\textbf{What are the drawbacks of the ICP algorithm?}\\\\
\textbf{How do you think the ICP algorithm can be improved, beside the techniques mentioned in ``Efficient variants of the ICP algorithm.'', in terms of efficiency and accuracy?}

\section{Assignment 2}
\subsection{Algorithm outline}
Outline here.
\subsection{Eliminate of points detected in the background}
Explanation here.
\subsection{Analysis}
\begin{itemize}
\item speed differences
\item Something about the determinant, rank of fundamental matrix had some property rank 1 - singular
\end{itemize}
\subsection{Chaining}
\section{Assignment3}
\subsection{Structure from motion}
Does the reconstruction has an ambiguity? Yes, probably. We did not eliminate this. Probably affine ambiguity. 
\subsection{Building 3D model}

\end{document}