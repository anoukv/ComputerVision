
\documentclass[12pt]{amsart}
\usepackage{geometry} % see geometry.pdf on how to lay out the page. There's lots.
\geometry{a4paper} % or letter or a5paper or ... etc
% \geometry{landscape} % rotated page geometry

% See the ``Article customise'' template for come common customisations

\title{Computer Vision}
\author{Anouk Visser and R\'emi de Zoeten}
\date{} % delete this line to display the current date

%%% BEGIN DOCUMENT
\begin{document}

\maketitle
%\tableofcontents

\section{Assignment 1}
\subsection{ICP}
There are different ways to improve the efficiency and the accuracy of ICP. Some of these techniques are discussed in ``Efficient variants of the ICP algorithm.''. In this assignment, students should also analyze various aspects such as accuracy, speed, stability and tolerance to noise by changing the point selection technique. Using all the points, uniform sub-sampling, random sub-sampling in each iteration and sub-sampling more from informative regions can be used as the point selection technique. Students are expected to implement these variants and report their findings in the final report.
\subsection{Merging scenes}
Estimate the camera poses using each two consecutive frames of given data. Using the estimated camera poses, merge the point-clouds of all the scenes into one point-cloud and visualize the result. 
\\\\
\textbf{Does the merging produce sufficient result? Discuss why.}
\\\\
\textbf{Now, estimate the camera pose and merge the results using every 2nd, 4th, and 10th frames. Does the camera pose estimation change?}
\\\\
Iteratively merge and estimate the camera poses for the consecutive frames (point-clouds). Supposed there are N frames (frame1, frame2 ,. . ., frameN). First, estimate the camera pose of frame2 (target) from frame1(base). Next, merge frame1 and frame2 into frame1,2. Then, estimate the camera pose of frame3(target) from frame1,2(base) and use it to merge frame1,2 and frame3 into frame1,2,3 and so on until reaching the final frame (frameN ).\\\\ \textbf{Do the estimated camera poses change in comparison with the previous estimates (Section 2.1)? Does this estimation produce better results?}
\subsection{Questions}
\textbf{What are the drawbacks of the ICP algorithm?}\\\\
\textbf{How do you think the ICP algorithm can be improved, beside the techniques mentioned in ``Efficient variants of the ICP algorithm.'', in terms of efficiency and accuracy?}

\section{Assignment 2}
\subsection{Fundamental Matrix Estimation}
To estimate the fundamental matrix $F$ we implemented the eight-point algorithm, the normalized eight-point algorithm and the normalized eight-point algorithm with RANSAC. Before we can estimate the fundamental matrix we need a set of corresponding points across two images. Therefore we first detect interest points in each image and extract their descriptors in order to match the interest points. We discard all matches that are in the background using the method described in section \ref{background}. The matches can then be used to estimate the fundamental matrix. \\\\
To check the correctness of the fundamental matrix estimation algorithms we verified that the fundamental matrix was singular ($\textit{det}(M) = 0$ for singular matrix $M$ ). We present our analysis of the estimation of the fundamental matrix for the first and second teddy-bear images in table \ref{fund}.
\begin{table}
    \begin{tabular}{l|l|l|l}
    ~                  & ep & nep & nepR \\ \hline
    F                  
    & $\bigl(\begin{smallmatrix}0.0000&0.0000&-0.0019\\ 0.0000&0.0000&-0.0009\\-0.0029&-0.0024&1.0000\end{smallmatrix} \bigr)$ 
    & $\bigl(\begin{smallmatrix}0.0000&0.0000&-0.0112\\ -0.0001&0.0000&-0.0001\\0.0222&-0.0012&-1.4995\end{smallmatrix} \bigr)$                      
    & $\bigl(\begin{smallmatrix}0.0000&0.0002&-0.0438\\ -0.0002&0.0000&0.0575\\0.0562&-0.0470&-3.2704\end{smallmatrix} \bigr)$                                  \\ \hline
    det(F)             & 5.3517e-28  & 1.5718e-24             & -1.3839e-22                        \\ \hline
    elapsed time (sec) & 0.000807    & 0.151748               & 0.833404                           \\
    \end{tabular}
    \label{fund}
    \caption{Fundamental matrix $F$, $\textit{det}(F)$ and the elapsed time in seconds to calculate $F$ for the eight-point algorithm (ep), normalized eight-point algorithm (nep) and the normalized eight-point algorithm with RANSAC (nepR)}
\end{table}
\subsection{Eliminate of points detected in the background}
\label{background}
To eliminate points in the background we ask for some user-input. The user specifies an initial rectangle that contains the object of interest. We then initialize MATLAB's activecontour method with a mask with the size of the user-specified rectangle. This gives us a more precise mask. We again define a rectangle based on the output of the contour which fits the object of interest more tightly than the user-specified rectangle. Before we start computing matches between the two images, we first discard all interested points that lie outside of this rectangle. 
\subsection{Chaining}
\section{Assignment3}
\subsection{Structure from motion}
\subsubsection{Point tracking}
By using our point-tracking algorithm as described in section \ref{non-existent} we do not find enough point to recover the structure. MENTION SOMETHING ABOUT THE AVERAGE NUMBER OF FRAMES A POINT IS FOUND IN. We could improve our point-tracking algorithm by using dense SIFT in order to obtain more points that can be matched across the images. However, a consequence of using dense SIFT is that it is computationally more expensive than using just detected interested points and their descriptors. \\\\
\textbf{Does the reconstruction has an ambiguity?} \\
Yes, the reconstruction suffers from affine ambiguity. When rotating the reconstruction so that the actual perpendicular and parallel lines look perpendicular and parallel we can recognize the house clearly. However, this only happens at a specific viewpoint, when viewing the reconstruction differently it is clear that lines that are parallel and lines that are perpendicular have not been reconstructed that way (this can be seen in image \ref{non-existent}). \subsection{Building 3D model}

\end{document}